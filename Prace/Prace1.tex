             
%%%%%%%%%%%%%%%%%%%%%%%%%%%%%%%%%%%%%%%%%%%%%%%%%%%%%%%%%%%%%%%%%%%%
%% I, the copyright holder of this work, release this work into the
%% public domain. This applies worldwide. In some countries this may
%% not be legally possible; if so: i~grant anyone the right to use
%% this work for any purpose, without any conditions, unless such
%% conditions are required by law.
%%%%%%%%%%%%%%%%%%%%%%%%%%%%%%%%%%%%%%%%%%%%%%%%%%%%%%%%%%%%%%%%%%%%

\documentclass[
  digital, %% This option enables the default options for the
           %% digital version of a~document. Replace with `printed`
           %% to enable the default options for the printed version
           %% of a~document.
  table,   %% Causes the coloring of tables. Replace with `notable`
           %% to restore plain tables.
  nolof,     %% Prints the List of Figures. Replace with `nolof` to
           %% hide the List of Figures.
  nolot,     %% Prints the List of Tables. Replace with `nolot` to
           %% hide the List of Tables.
  %% More options are listed in the user guide at
  %% <http://mirrors.ctan.org/macros/latex/contrib/fithesis/guide/mu/fi.pdf>.
]{fithesis3}
%% The following section sets up the locales used in the thesis.
\usepackage[resetfonts]{cmap} %% We need to load the T2A font encoding
\usepackage[T1,T2A]{fontenc}  %% to use the Cyrillic fonts with Russian texts.
\usepackage[
  main=english, %% By using `czech` or `slovak` as the main locale
                %% instead of `english`, you can typeset the thesis
                %% in either Czech or Slovak, respectively.
  german, russian, czech, slovak %% The additional keys allow
]{babel}        %% foreign texts to be typeset as follows:
%%
%%   \begin{otherlanguage}{german}  ... \end{otherlanguage}
%%   \begin{otherlanguage}{russian} ... \end{otherlanguage}
%%   \begin{otherlanguage}{english}   ... \end{otherlanguage}
%%   \begin{otherlanguage}{slovak}  ... \end{otherlanguage}
%%
%% For non-Latin scripts, it may be necessary to load additional
%% fonts:
\usepackage{paratype}
\def\textrussian#1{{\usefont{T2A}{PTSerif-TLF}{m}{rm}#1}}
%%
%% The following section sets up the metadata of the thesis.
\thesissetup{
    date          = \the\year/\the\month/\the\day,
    university    = mu,
    faculty       = fi,
    type          = mgr,
    author        = Matěj Vaněk,
    gender        = m,
    advisor       = {doc. Mgr. Radek Pelánek, Ph.D.},
    title         = {Teacher's Dashboard for Introductory Programming Educational Systems},
    TeXtitle      = {Teacher's Dashboard for Introductory Programming Educational Systems},
    keywords      = {introductory programming},
    TeXkeywords   = {introductory programming},
}

\thesislong{abstract}{Text of my abstract
}

\thesislong{thanks}{
    I would like to thank my supervisor, doc. Mgr. Radek Pelánek, Ph.D., and Mgr. Tomáš Effenberger. Also my parents and Zuzana Seménková.
}

%% The following section sets up the bibliography.
\usepackage{csquotes}
\usepackage[              %% When typesetting the bibliography, the
  backend=biber,          %% `numeric` style will be used for the
  style=numeric,          %% entries and the `numeric-comp` style
  citestyle=iso-numeric, %% for the references to the entries. The
  sorting=none,           %% entries will be sorted in cite order.
  sortlocale=en_US,         %% For more unformation about the available
  bibencoding=UTF8,
  babel=other
]{biblatex}               %% `style`s and `citestyles`, see:
%% <http://mirrors.ctan.org/macros/latex/contrib/biblatex/doc/biblatex.pdf>.
\addbibresource{Literatura.bib} %% The bibliograpic database within
                          %% the file `example.bib` will be used.
\usepackage{makeidx}      %% The `makeidx` package contains
\makeindex                %% helper commands for index typesetting.
%% These additional packages are used within the document:
\usepackage{paralist}
\usepackage{amsmath}
\usepackage{amsthm}
\usepackage{amsfonts}
\usepackage{url}
\usepackage{menukeys}

\makeatletter\thesis@load
\makeatletter
\def\thesis@blocks@thanks{%
\ifx\thesis@thanks\undefined\else
\thesis@blocks@clear
\begin{alwayssingle}%
\chapter*{\thesis@@{thanksTitle}}%
\thesis@thanks
\end{alwayssingle}%
\fi}
\makeatother

\begin{document}


\chapter{Shrnutí práce}
Minulý semestr:
\begin{enumerate}
	\item Graf četnosti slov v jednotlivých úlohách
	\item Distance matrix vzorových řešení podle
	\begin{enumerate}
		\item bag of words
		\item bag of words binární
		\item bag of words logaritmované
		\item bag of words tf-idf
		\item levenshteinova vzdálenost
		\item vlastní tree edit distance
	\end{enumerate}
	\item Korelace výše zmíněných metod podle
	\begin{enumerate}
		\item pearsona
		\item spearmana
		\item prumerny prunik na top 5 nejpodobnejsich ulohach pro kazdou ulohu
		\item prumerny prunik na top 20 nejpodobnejsich ulohach pro kazdou ulohu
	\end{enumerate}
	\item Korelace (pearson) výše zmíněných metod
	\item Výskyt slov v úlohách (feature matrix) podle
	\begin{enumerate}
		\item bag of words
		\item bag of words binární
		\item bag of words logaritmované
		\item bag of words tf-idf
	\end{enumerate}
	\item Korelace (??) výskytu slov v úlohách (feature matrices) podle
	\begin{enumerate}
		\item bag of words
		\item bag of words binární
		\item bag of words logaritmované
		\item bag of words tf-idf
	\end{enumerate}
\end{enumerate}

Tento semestr:
\begin{enumerate}
	\item Korelace času a operací po levelech
	\item Histogramy studentů podle úloh
	\begin{enumerate}
		\item časů
		\item operací
		\item poměru operace-čas
		\item mazání
		\item podíl častých stavů
	\end{enumerate}
	\item Scatter ploty času a editů (logaritmované)
	\begin{enumerate}
		\item obyč
		\item nelogaritmované
		\item ořezaná delta na 30 s
		\item obyč s zafixovaným obarvením podle mediánů na dané úloze
	\end{enumerate}
	\item Trajektorie uživatelů skrz průchody (UDĚLAT I JEN NA ÚSPĚŠNÝCH)
	\item Predikce do Ne/Pomalu/Rychle a Pomalu/Rychly 
		

\end{enumerate}


\chapter{Úvod}



\chapter{Závěr}


\nocite{*}

{\csname captions\languagename\endcsname %% Temporarily override
%% the BibLaTeX localization with the original babel definitions.
\makeatletter %% Use the correct localization of the quotations.
  \thesis@selectLocale{\thesis@locale}\makeatother
\printbibliography[heading=bibintoc]} %% Print the bibliography.



\appendix %% Start the appendices.


\end{document}